

\documentclass{acm_proc_article-sp}


\usepackage{graphicx}
\usepackage{listing}
\usepackage{listings} \lstset{basicstyle=\tiny,numbers=none, breaklines=true, numberstyle=\tiny, numbersep=5pt,firstnumber=last,language=XML,escapeinside={(*@}{@*)} }  

% *** CITATION PACKAGES ***
%
\usepackage{cite}
\usepackage{url}
%\usepackage[scaled=point85]{luximono}
\hyphenation{op-tical net-works semi-conduc-tor name-space}

\begin{document}

\title{VIENNA Add-In \\ Visualizing Inter-ENterprise Network Architectures}


\numberofauthors{2} %  in this sample file, there are a *total*
% of EIGHT authors. SIX appear on the 'first-page' (for formatting
% reasons) and the remaining two appear in the \additionalauthors section.
%
\author{
% You can go ahead and credit any number of authors here,
% e.g. one 'row of three' or two rows (consisting of one row of three
% and a second row of one, two or three).
%
% The command \alignauthor (no curly braces needed) should
% precede each author name, affiliation/snail-mail address and
% e-mail address. Additionally, tag each line of
% affiliation/address with \affaddr, and tag the
% e-mail address with \email.
%
% 1st. author
\alignauthor
Christian Huemer, Philipp Liegl, Thomas Motal, Rainer Schuster, Marco Zapletal \\
       \affaddr{Vienna University of Technology}\\
       \affaddr{Favoritenstrasse 9-11/188}\\
       \affaddr{1040 Vienna, Austria}\\
       \email{\{firstname.lastname\}@tuwien.ac.at}       
\alignauthor
Christian Eis, Martina Hiesinger, Fabian Kromer, Robert Kromer, Andreas Kuntner, Christian Pichler, Michael Strommer\\
       \affaddr{Research Studios Austria}\\
       \affaddr{Thurngasse 8/3/20}\\
       \affaddr{1090 Vienna, Austria}\\
       \email{office.ios@researchstudio.at}       
% 2nd. author
%\alignauthor
%Christian Pichler\\
%       \affaddr{Research Studios Austria}\\
%       \affaddr{Thurngasse 8/3/20}\\
%       \affaddr{1090 Vienna, Austria}\\
%       \email{cpichler@researchstudio.at}
}

\date{28 April 2009}

\maketitle
\begin{abstract}

%What is the problem
The definition of concise and interoperable business documents has become one of the key issues in today's electronic business transactions. Usually business document definitions are defined on a conceptual level and logical level artifacts such as XML schema definitions are derived from it. Appropriate tool support and well agreed standards for such a scenario are still rather low.
In this paper we present our tool VIENNA Add-In which supports a business document modeler in creating Core Component compliant business document model using the Unified Modeling Language (UML). The core components standard is maintained by UN/CEFACT (United Nations Center for Trade Facilitation and Electronic Business) and defines reusable building blocks for constructing business documents. 
Our tool provides a set of powerful features such as model validation, semi-automatic generation of model artifacts, and generation of fully compliant XML schema definitions from a conceptual model representation. 


%Why is it a problem


%What is the solution


%What is the benefit


\end{abstract}

% A category with the (minimum) three required fields
\category{H.4}{Information Systems Applications}{Miscellaneous}


\terms{Business document modeling, conceptual modeling, XML schema generation}




\section{Introduction}


\begin{figure}
 \centering
   \includegraphics[width=0.37\textwidth]{figures/addinoverview.pdf}
 \caption{VIENNA Add-In overview}
 \label{fig:viennaaddinoverview}
\end{figure}


\section{UML Profile Definition}

\section{Semi-automatic artifact generation}

\section{XML schema generation}


\cite{man:umm2} \cite{man:upcc} \cite{man:VIENNAAddIn}





\bibliographystyle{abbrv}
\bibliography{references}  
\end{document}
